% \documentclass[9pt,oneside]{amsart}
\usepackage{multicol}
% \usepackage[a4paper,
%             width=170mm,
%             top=18mm,
%             bottom=22mm,
%             includeheadfoot]{geometry}
% \usepackage[bookmarks=true,
%             unicode=true,
%             pdftitle={ACTUS Technical Specification},
%             pdfauthor={ACTUS Financial Research Foundation},
%             pdfkeywords={ACTUS, Financial Contracts, Algorithmic Contracts, Technical Specification},
%             pdfborder={0 0 0.5 [1 3]}]{hyperref}

% ---------------------- floats ----------------------
\usepackage{graphicx}
\usepackage{float}
\usepackage{longtable}

% ---------------------- math ----------------------
\usepackage{amsmath}
\usepackage{amssymb}
\usepackage{amsthm}
\newtheorem{example}{Example}

% ---------------------- custom tables ----------------------
\newenvironment{states}[1]{
	\hfill % force subsection before longtable
	\begin{longtable}{| p{0.05\textwidth} | p{0.48\textwidth} |  p{0.43\textwidth} |}
	\multicolumn{3}{c}{\textbf{#1: State Variables Initialization}}\\ \hline
	\textbf{State} & \textbf{Initialization per $t_0$} & \textbf{Comments} \\
	\hline
	\endfirsthead
	\multicolumn{3}{c}{\textit{Continua desde la página anterior}} \\
	\hline
	\textbf{State} & \textbf{Initialization per $t_0$} & \textbf{Comments} \\
	\hline
	\endhead
	\hline \multicolumn{3}{r}{\textit{Continua en la siguiente página}} \\
	\endfoot
	\endlastfoot
}{%
	\hline
	\end{longtable}
}

\newenvironment{schedule}[1]{
	\hfill % force subsection before longtable
	\begin{longtable}{| p{0.05\textwidth} | p{0.5\textwidth} |  p{0.4\textwidth} |}
	\multicolumn{3}{c}{\textbf{#1: Contract Schedule}}\\
	\hline
	\textbf{Event} & \textbf{Schedule} & \textbf{Comments} \\
	\hline
	\endfirsthead
	\multicolumn{2}{c}{\textit{Continua desde la página anterior}} \\
	\hline
	\textbf{Event} & \textbf{Schedule} & \textbf{Comments} \\
	\hline
	\endhead
	\hline \multicolumn{2}{r}{\textit{Continua en la siguiente página}} \\
	\endfoot
	\endlastfoot
}{%
	\hline 
	\end{longtable}
}

\newenvironment{functions}[1]{
	\hfill % force subsection before longtable
    	\begin{longtable}{| p{0.05\textwidth} | p{0.42\textwidth} |  p{0.48\textwidth} |}
	\multicolumn{3}{c}{\textbf{#1: State Transition Functions and Payoff Functions}}\\
	\hline
	\textbf{Event} & \textbf{Payoff Function} & \textbf{State Transition Function}\\
	\hline
	\endfirsthead
	\multicolumn{2}{c}{\textit{Continua desde la página anterior}} \\
	\hline
	\textbf{Event} & \textbf{Payoff Function} & \textbf{State Transition Function}\\
	\hline
	\endhead
	\hline \multicolumn{2}{r}{\textit{Continua en la siguiente página}} \\
	\endfoot
	\endlastfoot
}{%
	\hline
        \end{longtable}
}


% ---------------------- custom notation ----------------------
\newcommand{\Real}{\mathbb{R}}
\newcommand{\Nat}{\mathbb{N}}
\newcommand{\svar}[2]{\textbf{#1}_{#2}}
\newcommand{\attr}[1]{\texttt{#1}}
\newcommand{\stf}[2]{STF\_#1\_#2()}
\newcommand{\pof}[2]{POF\_#1\_#2()}
\newcommand{\dfl}[1]{D (\textbf{Prf}_{#1})}
\newcommand{\sgn}{R (\attr{CNTRL})}
\newcommand{\sdl}[3]{S (#1,#2,#3)}
\newcommand{\sdll}[4]{S (#1,#2,#3,#4)}
\newcommand{\vsdl}[3]{\vec{S} (#1,#2,#3)}
\newcommand{\yfr}[2]{Y (#1,#2)}
\newcommand{\yfrfunc}{Y}
\newcommand{\ann}[5]{A (#1,#2,#3,#4,#5)}
\newcommand{\annfunc}{A}
\newcommand{\obs}[3]{O^{#1} (#2,#3)}
\newcommand{\obsfull}[5]{O^{#1} (#2,#3,#4,#5)}
\newcommand{\obsfunc}[1]{O^{#1}}
\newcommand{\cldev}[3]{U^{ev} (#1,#2 \mid\{#3\})}
\newcommand{\cldsv}[4]{U^{sv} (#1,#2,\svar{#3}{} \mid\{#4\})}
\newcommand{\cldsvs}[3]{U^{sv} (#1,#2,\svar{#3}{})}
\newcommand{\cldca}[2]{U^{ca} (#1,#2)}
\newcommand{\cldfunc}[1]{U^{#1}}
% \newcommand{\undef}{\varnothing}
\newcommand{\tmax}{t^{max}}
\newcommand{\tev}[1]{\tau(#1)}
\newcommand{\fev}[1]{f (#1)}
\newcommand{\payoff}[2]{F (#1,#2)}


% ---------------------- misc ----------------------
\usepackage{verbatim}
% \usepackage{natbib}
\setlength\parindent{0pt}

