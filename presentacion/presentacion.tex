\documentclass{beamer}

\mode<presentation>
{
  \usetheme{Rochester}
  %\useoutertheme{infolines}
  %\usecolortheme[RGB={125,173,51}]{structure}
  % or ...
  %\setbeamercovered{transparent}
  % or whatever (possibly just delete it)
}

\usepackage[spanish]{babel}
\usepackage[utf8]{inputenc}
\usepackage{times}
\usepackage{multicol}
\usepackage{verbatim} 
\usepackage{fancyvrb}
\graphicspath{{images/}}
\usepackage{listings}
\usepackage{tikz}
\usetikzlibrary{arrows}
\usetikzlibrary{shapes}
\tikzstyle{block}=[draw opacity=0.7,line width=1.4cm]
\usepackage{hyperref}
\usepackage{xcolor}
\usepackage[breakable,listings,skins,hooks]{tcolorbox}
\hypersetup{
  colorlinks,
  allcolors=.,
  urlcolor=blue,
}

\usefonttheme[onlymath]{serif}
%\lstdefinelanguage{Marlowe}{%
  language     = Haskell,
  morekeywords = {Close, Pay, Assert, If, When, Let},
}
\lstdefinelanguage{Isabelle}{%
  language     = ML,
  morekeywords = {theory, imports, begin, end},
}

\definecolor{codegreen}{rgb}{0,0.6,0}
\definecolor{codegray}{rgb}{0.5,0.5,0.5}
\definecolor{codepurple}{rgb}{0.58,0,0.82}
\definecolor{backcolour}{rgb}{0.95,0.95,0.92}

\lstdefinestyle{Haskell-cardano}{
    backgroundcolor=\color{backcolour},   
    commentstyle=\color{codegreen},
    keywordstyle=\color{magenta},
    numberstyle=\tiny\color{codegray},
    stringstyle=\color{codepurple},
    basicstyle=\ttfamily\footnotesize,
    breakatwhitespace=false,         
    language=Haskell,
    breaklines=true,                 
    captionpos=b,                    
    keepspaces=true,                 
    numbers=none,                    
    numbersep=5pt,                  
    showspaces=false,                
    showstringspaces=false,
    showtabs=false,                  
    tabsize=2
}

\definecolor{isarblue}{HTML}{006699}
\definecolor{isargreen}{HTML}{009966}
\lstdefinelanguage{isabelle}{%
    keywords=[1]{type_synonym,datatype,fun,abbreviation,definition,proof,lemma,theorem, theory,corollary},
    keywordstyle=[1]\bfseries\color{isarblue},
    keywords=[2]{where,assumes,shows,and, imports, begin, end},
    keywordstyle=[2]\bfseries\color{isargreen},
    keywords=[3]{if,then,else,case,of,SOME,let,in,O},
    keywordstyle=[3]\color{isarblue},
}
\lstdefinestyle{Isabelle}{%
  language=isabelle,
  escapeinside={&}{&},
  columns=fixed,
  extendedchars,
  frame=single,
  basewidth={0.5em,0.45em},
  basicstyle=\ttfamily,
  mathescape,
}


\title[Verificación de smart contracts en Marlowe]% (optional, use only with long paper titles)
{Verificación de smart contracts en Marlowe para la blockchain Cardano}

\author[Julián Ferres] % (optional, use only with lots of authors)
{~Julián Ferres}
% - Give the names in the same order as the appear in the paper.
% - Use the \inst{?} command only if the authors have different
%   affiliation.
\institute[FIUBA] % (optional, but mostly needed)
{
  %\inst{1}
  Facultad de Ingeniería\\Universidad de Buenos Aires.
}
\date{\today}

% Acá se puede insertar el logo de la UBA
\pgfdeclareimage[height=0.75cm]{fiuba}{images/FIUBA_Logo.png}
\logo{\pgfuseimage{fiuba}}


% Tamaño de fuente en la primer página
\setbeamerfont{author}{size=\small}
\setbeamerfont{institute}{size=\scriptsize}
\setbeamerfont{date}{size=\scriptsize}

% Delete this, if you do not want the table of contents to pop up at
% the beginning of each subsection:
%\AtBeginSubsection[]
\AtBeginSubsection[]
{
  \begin{frame}{Índice de contenidos}
  \footnotesize
%  \begin{multicols}{2} 
    \tableofcontents[currentsection, currentsubsection]
%  \end{multicols}
  \end{frame}
}


\begin{document}
\pgfdeclarelayer{background}
\pgfsetlayers{background,main}

\begin{frame}
	\titlepage
\end{frame}


\begin{frame} 
	\footnotesize
	\frametitle{Índice de Contenidos}
	%\begin{multicols}{2} 
	\tableofcontents
	%\end{multicols}
\end{frame}

%# Esquema de temas para la presentación
%
%- Introducción
%    - Similar a lo contado en el primer capítulo del documento. Una breve descripción de que son, por que son importantes y las opciones de IOHK para:
%        - Blockchains
%        - Criptomonedas
%        - Smart contracts
%
%    - Un poco de introducción a ACTUS, mencionar porque es importante y útil dicho estandar (mencionando que Cardano planea implementar todos los contratos en el fúturo)
%        - Describir un poco el módelo que se utiliza (Scheduling, estado, inicialización de variables de estado, POFs y STFs) y mostrar algún fragmento de la tabla.
%
%    - Introducción a pruebas:
%        - Por qué es importante verificar contratos
%        - Algunas variantes y porque se utilizó Isabelle en Particular
%        - Un poco de descripción breve de Isabelle
%
%- Mostrar la escritura de contratos
%    - En esta sección me enfocaría más en describir lo necesario para escribir un contrato en Cardano, y quizas agregue algún snippet corto de alguna parte de los contratos que implementé. Esto es porque realmente es requerimiento entender el módelo del generador para empezar a ver como programar un nuevo contrato.
%
%
%- Mostrar y describir algunas de las pruebas:
%    - Va a ser necesario al menos una mención al modelo de Marlowe y las funciones más importantes implementadas en Isabelle, sino no se va a saber la utilidad de las funciones en las que estamos probando propiedades.
%    - Depende de las pruebas que decidamos mostrar, quizás mencionar un poco la motivación y técnicas de prueba que fui usando. 
%
%
%- Conclusión y cierre, hacer un breve comentario de los temas abordados y posibles fuentes de desarrollo que se desprenden de la tesis.

\section{Introducción}

\subsection{Blockchains, Criptomonedas y Smart contracts}

\begin{frame}

\end{frame}

\subsection{ACTUS}

\begin{frame}

\end{frame}

\subsection{Verificación formal}

\begin{frame}

\end{frame}

\section{Escribiendo contratos ACTUS en Cardano}

\subsection{Contratos en Cardano}

\begin{frame}

\end{frame}

\section{Verificando propiedades en contratos en Marlowe}

\subsection{El modelo de Marlowe}

\subsection{Pruebas sencillas sobre contratos específicos}

\subsection{Warnings en Auction}

\section{Conclusión}

\subsection{Resumen}

\subsection{Posibles temas de desarrollo futuro}



\section{Bibliografía}

\begin{frame}{Bibliografía}
	
\end{frame}

\end{document}
